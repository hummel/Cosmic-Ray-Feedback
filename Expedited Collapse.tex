\subsubsection{Expedited Collapse}
The CR background also works to heat the gas.  In concert with the boosted $\htwo$ fraction, this lowers the density threshold for efficient cooling, expediting the collapse.  This is demonstrated in \ref{fig:efrac}, where we see the density at which the gas begins to cool drop as the CR background strength increases.  This necessarily expedites the subsequent phases of the collapse, as seen clearly in Figure \ref{fig:collapse}, where we show the total gas mass within 1 pc (physical) of the minihalo's centre as a function of redshift.  The impact this has on the environment in which the minihalo collapses is see in Figure \ref{fig:structure}, where we show the final simulation output on various scales for Halo 1 and Halo 2 both in the absence of any CR background ($\ucr=0$) and with $\ucr=10^5\,u_0$. Low density filamentary gas is heated and prevented from collapsing, while gas above a $\ucr$-dependent density threshold experiences enhanced cooling and thus has its collapse expedited.  As a result, the dynamical environment in which the first sink particles form varies significantly depending on the CR background.

