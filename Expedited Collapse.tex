\subsubsection{Expedited Collapse}
\label{sec:expedited_collapse}
Boosting the $\htwo$ fraction in the gas lowers the density threshold for efficient cooling, allowing the gas to fulfill the Rees-Ostriker (\citeyear{ReesOstriker1977}) criterion sooner by driving the cooling time $t_{\rm \small cool}$ below the freefall time $t_{\rm \small ff}$.
This is demonstrated in Figure \ref{fig:efrac}, where we see the density at which the gas begins to cool drop as the CR background strength increases.  
This necessarily expedites the subsequent phases of the collapse, as seen clearly in Figure \ref{fig:collapse}, where we show the total gas mass within 1 pc (physical) of the minihalo's centre as a function of redshift.  
The impact this has on the environment in which the minihalo collapses is seen in Figure \ref{fig:structure}, where we show the final simulation output on various scales for Halo 1 and Halo 2, both in the absence of any CR background ($\ucr=0$) and with $\ucr=10^5\,u_0$. 
Low density filamentary gas is heated and prevented from collapsing, reducing the clumping of the IGM and possibly impacting the early stages of reionisation.
Gas above a $\ucr$-dependent density threshold on the other hand experiences enhanced cooling and thus has its collapse expedited.
As a result, the dynamical environment in which the first sink particles form varies significantly depending on the strength of the CR background.