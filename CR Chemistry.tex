\subsection{Cosmic Ray Ionisation and Heating}
\label{CRchem}
To study the impact of an ionising CR background on primordial star formation, we implement a uniform CR background as discussed in \RefSec{crb}. Accounting for attenuation of the CRB as discussed in \RefSec{attenuation}, the heating rate $\Gamma_{\rm \small CR}$ is given by
\begin{equation}
\Gamma_{\rm \small CR} = n_{\rm \small H} E_{\rm \small heat} k_{\rm \small CR}.
\end{equation}
where $n_{\rm \small H}$ is the number density of hydrogen and $E_{\rm \small heat}$ is the energy deposited as heat per interaction.

, such that the ionisation rate coefficient $k_{\rm \small CR}$ for atomic hydrogen \citep{Schlickeiser2002} can be written as
\begin{equation}
k_{\rm \small CR} = \frac{1.82\times10^{-7}\,{\rm \small eV\,s}^{-1}}{50\,{\rm \small eV}} 
    \int_{\epsilon_{\rm min}}^{\epsilon_{\rm max}} f(\epsilon) \frac{dn_{\rm \tiny CR}}{d\epsilon} d\epsilon,
\end{equation}
where $\epsilon_{\rm min} = 10^6\ev$, $\epsilon_{\rm max}= 10^{15}\ev$.

While CRs lose about $50\ev$ per interaction, only about $6\ev$ of that goes towards heating in a neutral medium \citep{SpitzerScott1969, ShullvanSteenberg1985}, such that the heating rate 