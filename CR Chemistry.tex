\subsection{Cosmic Ray Ionisation and Heating}
\label{CRchem}
Over the length scale of our simulation box, the IGM is optically thin to cosmic rays above $10^6\ev$. This allows us to implement a uniform CR background as discussed in \RefSec{context} by assuming attenuation of the incident radiation while penetrating the minihalo is negligible, such that the ionisation rate coefficient $k_{\rm \small CR}$ for atomic hydrogen \citep{Schlickeiser2002} can be written as
\begin{equation}
k_{\rm \small CR} = \frac{1.82\times10^{-7}\,{\rm \small eV\,s}^{-1}}{50\,{\rm \small eV}} 
    \int_{\epsilon_{\rm min}}^{\epsilon_{\rm max}} f(\epsilon) \frac{dn_{\rm \tiny CR}}{d\epsilon} d\epsilon,
\end{equation}
where $\epsilon_{\rm min} = 10^6\ev$, $\epsilon_{\rm max}= 10^{15}\ev$,
\begin{equation}    
    f(\epsilon) = (1 + 0.0185 \,{\rm ln}\beta )\, \frac{2 \beta^2}{\beta_0^3 + 2 \beta^3}
\end{equation}
and
\begin{equation}
    \beta =  \sqrt{1 - \left( \frac{\epsilon}{m_{\rm \tiny H}c^2}+1 \right)^{-2}}.
\end{equation}
Here, $m_{\rm \small H}$ is the mass of hydrogen, $c$ is the speed of light, and $\beta = v/c,$. $\beta_0$ is the cutoff below which the interaction between CRs and the gas decreases sharply; we use $\beta_0=0.01$, appropriate for CRs traveling through a neutral IGM \citep{StacyBromm2007}.

While CRs lose about $50\ev$ per interaction, only about $6\ev$ of that goes towards heating in a neutral medium \citep{SpitzerScott1969, ShullvanSteenberg1985}, such that the heating rate $\Gamma_{\rm \small CR}$ is given by
\begin{equation}
\Gamma_{\rm \small CR} = n_{\rm \small H} E_{\rm \small heat} k_{\rm \small CR}.
\end{equation}
where $n_{\rm \small H}$ is the number density of hydrogen and $E_{\rm \small heat}$ is the energy deposited as heat per interaction.