\section{Cosmic Rays in the Early Universe}
\label{sec:context}
While there are several possible sources of CRs at high redshifts including primordial black holes, topological defects, supermassive particles, and structure formation shocks, the most likely source of CRs in the early universe is supernova (SN) explosions \citep[e.g.,][]{GinzburgSyrovatskii1969,BiermannSigl2001,Stanev2004,Pfrommeretal2006}, wherein CRs are accelerated by the SN shock wave via the first-order Fermi process.  In this scenario, high-energy particles diffuse back and forth across the shock wave, increasing their energy by a small percentage each time, and resulting in a differential spectrum of CR number density per energy \citep{Longair1994} of the form
\begin{equation}
    \frac{{\rm d}n_{\rm \small CR}}{{\rm d}\epsilon} = \frac{n_{\rm norm}}{\epsilon_{\rm min}}
    \left( \frac{\epsilon}{\epsilon_{\rm min}} \right)^{-2},
\end{equation}
where $n_{\rm \small CR}$ is the CR number density, $\epsilon$ is the kinetic energy of the CR, $\epsilon_{\rm min}$ is the low-energy cutoff of the CR spectrum, and $n_{\rm norm}$ is a normalising density factor. 

The value of $\epsilon_{\rm min}$ is crucial, as the ionisation cross-section of non-relativistic CRs varies rougly as $\epsilon^{-1}$ for $\epsilon \lesssim 10^5\ev$. Higher energy CRs thus travel much farther between interactions and are less able to effectively deposit their energy into the gas, sugh that the low-energy end of the CR spectrum provides the majority of the ionisation and heating.  While estimates of the lowest possible energy CR protons could gain in a SN shock wave vary significantly, below $10^5\ev$ the velocity of the CR drops below the approximate orbital velocity of electrons in the ground state of atomic hydrogen, and the interaction cross-section dimishes rapidly \citep{Schlickeiser2002}. 

We therefore employ $\epsilon_{\rm min} = 10^6\ev$ as the lower bound for the CR background in our simulations.  While CRs with energies below $10^6\ev$ may account for $\sim$$5-50$ percent of the CR energy budget, they also deposit a substantial fraction of their energy into the IGM \citep{SazonovSunyaev2015}. Consequently, they are unable to efficiently contribute to the build-up of a large-scale CR background. This neglects the contribution $\epsilon < 10^6\ev$ CRs make to the ionisation and heating of primordial gas; however, the impact is minor, as show by \citet{StacyBromm2007}. Using one-zone models, they found that extending the CR spectrum to as low as $\epsilon_{\rm min} = 10^3\ev$ altered the resulting ionisation and heating rates by less than a factor of two.  It should also be noted that the uncertainty in $\epsilon_{\rm min}$ is on par with that for the fraction of the SN energy going into CR production, $f_{\rm \small CR}$, which is expected to be $\sim$$10-20$ percent \citep{CaprioliSpitkovsky2014}.  Here we assume $f_{\rm \small CR} = 0.1$.

The upper limit of the CR energy spectrum $\epsilon_{\rm max}$ depends on the strength of the ambient magnetic field, as the Fermi acceleration process is linearly dependent of the magnetic field through which the SN shock wave propagates. While the strength, generation mechanism, and distribution of magnetic fields in the early universe remain highly uncertain \citep{DurrerNeronov2013}, recent observations have placed lower bounds on the modern-day intergalactic magnetic field strength ranging from $10^{-18}\,$G \citep{Dermeretal2011} to $3\times10^{-16}\,$G \citep{NeronovVovk2010}, while other estimates place the field strength as high as $10^{-15}\,$G \citep{AndroKusenko2010}.  Simply accounting for flux freezing, the magnetic field at redshift $z$ is given by
\begin{equation}
B(z) = B_0 (1+z)^2
\end{equation}
where $B_0$ is the magnetic field at the current epoch. The magnetic field strength in the IGM at $z=20$ then was likely in the range $4\times10^{-16}$ to $4\times10^{-13}\,$G.  Flux freezing and dynamo effects during minihalo collapse and subsequent star formation may then drive the local magnetic field a few orders of magnitude higher.  With this in mind we set $\epsilon_{\rm max} = 10^{15}\ev$.  So long as $\epsilon_{\rm max}$ is below the GZK cutoff $\epsilon_{\rm \small GZK}$ at that redshift, given by \citep{StacyBromm2007}
\begin{equation}
\epsilon_{\rm \small GZK} = \frac{5\times10^{19}\ev}{1+z},
\end{equation}
the precise value of $\epsilon_{\rm max}$ is not crucial, since the great majority of the heating and ionisation is provided by CRs near $\epsilon_{\rm min}$.


Normalising the differential CR spectrum over the aforementioned limits $\epsilon_{\rm min} - \epsilon_{\rm max}$ to the total energy density $\ucr$ in cosmic rays results in a CR energy spectrum increasing over cosmic time in the following fashion:
 \begin{equation}
 \frac{{\rm d}n_{\rm \small CR}}{{\rm d}\epsilon}(z) = \frac{u_{\rm \small CR}(z)}{\epsilon_{\rm min}^2{\rm ln}\,\epsilon_{\rm max}{\large /}\epsilon_{\rm min}}  \left( \frac{\epsilon}{\epsilon_{\rm min}} \right)^{-2},
 \end{equation}
where we estimate $\ucr$ as follows:
\begin{equation}
u_{\rm \small CR}(z) = f_{\rm \small CR} E_{\rm \small SN}\, f_{\rm \small SN} \Psi_{*}(z)\, t_{\rm \small H}(z) (1+z)^3,
\end{equation}
where $f_{\rm \small CR}$ is the fraction of the SN explosion energy $E_{\rm \small SN}$ going into CR production, $f_{\rm \small SN}$ is the mass fraction of stars formed which die as SNe, and $\Psi_{*}(z)$ is the comoving star formation rate density (SFRD) as a function of redshift.  The Hubble time $t_{\rm \small H}$ accounts for the time CRs have had to propagate through the universe since their creation, and the factor of $(1+z)^3$ accounts for the conversion from a comoving SFRD to a physical energy density. As in \citet{Hummeletal2015}, we base our estimate of $u_{\rm \small CR}(z)$ on the Pop III SFRD calculated by \citet{GreifBromm2006}, but see \citet{Campisietal2011} for a more recent calculation. 

The mass fraction of Pop III stars dying as SNe depends strongly on their IMF, and while the complex physical processes at play in gas collapsing from IGM to protostellar densities have so far prevented a definitive answer to this question, the emerging consensus is that the Pop III IMF was somewhat top-heavy with a characteristic mass of $\sim$ a few $\times 10\msun$ \citep{Bromm2013}.  Given this, we assume one SN is produced for every $50\msun$ of stars formed and each SN-producing star dies quickly as a core-collapse SN with $E_{\rm \small SN} = 10^{51}\erg$, 10 percent of which goes into CR production.  

This fiducial estimate for $\ucrz$---henceforth referred to as model $u_0$---is shown in Figure \ref{fig:ucr}, where we compare the energy density in CRs to several other components of the high-redshift IGM, including the energy density of the cosmic microwave background, the gas thermal energy density, and the range of estimates for the energy density in the magnetic field, as well as the most conservative upper limits placed on $\ucr$ by \citet{RollindeVangioniOlive2006}. Given the huge uncertainties inherent in estimating the strength of the high-$z$ CR background, we also consider five additional models with 10, $10^2$, $10^3$, $10^4$, and $10^5$ times the energy density of model $u_0$, as shown in Figure \ref{fig:ucr}.
 