\section{Introduction}
\label{intro}

With the upcoming launch of the James Webb Space Telescope (JWST), the era of the first galaxies will be revealed, allowing us to bridge the gap between detailed ab-initio cosmological simulations and direct observations for the first time. While the first stars themselves are unlikely to be directly observable, the play a major role at this critical juncture in cosmic evolution, setting the stage not only for all subsequent episodes of star formation, but for the formation of the first galaxies themselves \citep{BarkanaLoeb2001,Miralda-Escude2003,Brommetal2009,Loeb2010, Bromm2013}.  As such, developing a thorough understanding of how these so-called Population III (Pop III) stars influenced their environment is crucial to developing a comprehensive picture of cosmic evolution. The basic picture of primordial star formation has been reasonably well established, with the consensus that the first stars formed at $z\gtrsim20$ in dark matter `minihaloes' having on the order of $10^5 - 10^6\msun$ \citep{CouchmanRees1986, HaimanThoulLoeb1996, Tegmarketal1997}. While pioneering numerical studies suggested that Pop III stars were very massive---on the order of $100\msun$---due to the lack of more efficient coolants than $\htwo$ \citep[e.g.,][]{BrommCoppiLarson1999, BrommCoppiLarson2002, AbelBryanNorman2002, Yoshidaetal2003, BrommLarson2004, Yoshidaetal2006, O'SheaNorman2007}, more recent simulations, aided by increased resolution, have found that significant fragmentation occurs during the star formation process \citep{StacyGreifBromm2010, Clarketal2011a, Clarketal2011b, Greifetal2011, Greifetal2012, StacyBromm2013, Hiranoetal2014}, leading to the emerging consensus that the Pop III initial mass function (IMF) was somewhat top-heavy with a characteristic mass of $\sim$ a few $\times 10\msun$ \citep{Bromm2013}.  The answer to this question is critical, as the extent to which the first stars influence their environment is largely controlled by their mass, which determines both their total luminosity and ionising radiation output \citep{Schaerer2002}, in addition to the details of their eventual demise \citep{Hegeretal2003, HegerWoosley2010, MaederMeynet2012}.

While the contributions Pop III stars make to chemical enrichment (\citealt{MadauFerraraRees2001, MoriFerraraMadau2002, BrommYoshidaHernquist2003, Hegeretal2003, UmedaNomoto2003, TornatoreFerraraSchneider2007, Greifetal2007, Greifetal2010, WiseAbel2008, Maioetal2011}; recently reviewed in \citealt{KarlssonBrommHawthorn2013}) and reionization \citep{Kitayamaetal2004, Sokasianetal2004, WhalenAbelNorman2004, AlvarezBrommShapiro2006, JohnsonGreifBromm2007, Robertsonetal2010} have been well studied, their impact on subsequent episodes of star formation is not as well understood.  As Pop III star formation primarily occurs in a predominantly neutral medium, the majority of their ionising output is absorbed, allowing only radiation well-removed from the Lyman-$\alpha$ transition to escape the immediate vicinity of the star-forming minihalo.  At the low energy end, far-ultraviolet radiation in the Lyman-Werner (LW) bands ($11.2 - 13.6\ev$) lacks sufficient energy to interact with atomic hydrogen, but can still effectively dissociate molecular hydrogen.  As $\htwo$ molecules serve as the primary coolant in primordial gas, this diminishes its ability to cool.  However, studies have found that the expected mean value of such radiation is far below the critical LW flux required to suppress $\htwo$ cooling \citep{Dijkstraetal2008}. At the high energy end, the neutral hydrogen cross section for X-rays and cosmic rays is small, allowing them to easily escape their host minihaloes.  We recently investigated the impact of a cosmic X-ray background on primordial star formation \citep{Hummeletal2014}; here we focus on the impact of a cosmic ray (CR) background.  

This paper is organized as follows: In \RefSec{context} we provide the cosmological context for this study, estimating the expected intensity of the CR background. Our numerical methodology is described in \RefSec{methods}, while our results are found in \RefSec{results}.  Finally, our conclusions are gathered in \RefSec{conclusions}.
Throughout this paper we adopt a $\Lambda$CDM model of hierarchical structure formation, using the following cosmological parameters: $\Omega_{\Lambda} = 0.7$, $\Omega_{\rm m} = 0.3$, $\Omega_{\rm B} = 0.04$, and $H_0 = 70 \kms \Mpc^{-1}$.


%The formation of the first stars marks a crucial turning point in cosmic history, signalling the end of the dark ages and initiating a significant increase in the complexity of cosmic evolution.  Understanding the formation of these so-called Population III (Pop III) stars is a crucial aspect of modern astrophysics and cosmology, as their emergence ushers in a fundamental transformation of the early universe \citep{BarkanaLoeb2001,Miralda-Escude2003,Brommetal2009,Loeb2010, Bromm2013}. The heavy elements forged in their cores and dispersed during the violent supernova explosions they produce initiated the chemical enrichment process (\citealt{MadauFerraraRees2001, MoriFerraraMadau2002, BrommYoshidaHernquist2003, Hegeretal2003, UmedaNomoto2003, TornatoreFerraraSchneider2007, Greifetal2007, Greifetal2010, WiseAbel2008, Maioetal2011}; recently reviewed in \citealt{KarlssonBrommHawthorn2013}), while the ionizing radiation and cosmic rays produced during their lives and subsequent deaths began the process of reionization \citep{Kitayamaetal2004, Sokasianetal2004, WhalenAbelNorman2004, AlvarezBrommShapiro2006, JohnsonGreifBromm2007, Robertsonetal2010}.  The extent to which the radiation and metal enrichment produced by the first stars influenced the intergalactic medium (IGM) and subsequent stellar generations, however, is largely determined by their mass.  The characteristic mass of Pop III stars determines their total luminosity and ionising radiation production \citep{Schaerer2002} as well as the specific details of the supernova explosions marking their demise \citep{Hegeretal2003, HegerWoosley2010, MaederMeynet2012}. As such, developing a thorough understanding of how the first stars are born, live, and die is crucial for developing a comprehensive picture of cosmic evolution.

