\section{Introduction}
\label{intro}

With the upcoming launch of the James Webb Space Telescope (JWST), the era of the first galaxies will be revealed, allowing us to bridge the gap between detailed ab-initio cosmological simulations and direct observations for the first time. While the first stars themselves are unlikely to be directly observable, they play a major role at this critical juncture in cosmic evolution, setting the stage not only for all subsequent episodes of star formation, but for the formation of the first galaxies themselves \citep{BarkanaLoeb2001,Miralda-Escude2003,Brommetal2009,Loeb2010,Bromm2013}.  As such, developing a thorough understanding of how these so-called Population III (Pop III) stars influenced their environment is crucial to developing a comprehensive picture of cosmic evolution. The basic aspects of primordial star formation have been reasonably well established, with the consensus that the first stars formed at $z\gtrsim20$ in dark matter `minihaloes' having on the order of $10^5 - 10^6\msun$ \citep{CouchmanRees1986,HaimanThoulLoeb1996,Tegmarketal1997}. 

While pioneering numerical studies suggested that Pop III stars were very massive---on the order of $100\msun$---due to the lack of more efficient coolants than $\htwo$ \citep[e.g.,][]{BrommCoppiLarson1999, BrommCoppiLarson2002, AbelBryanNorman2002, Yoshidaetal2003, BrommLarson2004, Yoshidaetal2006, O'SheaNorman2007}, more recent simulations, aided by increased resolution, have found that significant fragmentation occurs during the star formation process \citep{StacyGreifBromm2010,Clarketal2011a,Clarketal2011b,Greifetal2011,Greifetal2012,StacyBromm2013,Hiranoetal2014}, leading to the emerging consensus that the Pop III initial mass function (IMF) was somewhat top-heavy with a characteristic mass of $\sim$ a few $\times 10\msun$ \citep{Bromm2013}.  The answer to this question is critical, as the extent to which the first stars influence their environment is largely controlled by their mass, which determines both their total luminosity and ionising radiation output \citep{Schaerer2002}, in addition to the details of their eventual demise \citep{Hegeretal2003,HegerWoosley2010,MaederMeynet2012}.

While the contributions Pop III stars make to chemical enrichment (\citealt{MadauFerraraRees2001,MoriFerraraMadau2002,BrommYoshidaHernquist2003,Hegeretal2003,UmedaNomoto2003,TornatoreFerraraSchneider2007,Greifetal2007,Greifetal2010,WiseAbel2008,Maioetal2011}; recently reviewed in \citealt{KarlssonBrommHawthorn2013}) and reionization \citep{Kitayamaetal2004,Sokasianetal2004,WhalenAbelNorman2004,AlvarezBrommShapiro2006,JohnsonGreifBromm2007,Robertsonetal2010} have been well studied, their impact on subsequent episodes of star formation is not as well understood.  As Pop III star formation primarily occurs in a predominantly neutral medium, the majority of their ionising output is absorbed, allowing only radiation less energetic than the Lyman-$\alpha$ transition to escape the immediate vicinity of the star-forming minihalo.  At the low energy end, far-ultraviolet radiation in the Lyman-Werner (LW) bands ($11.2 - 13.6\ev$) lacks sufficient energy to interact with atomic hydrogen, but can still effectively dissociate molecular hydrogen.  As $\htwo$ molecules serve as the primary coolant in primordial gas, this diminishes its ability to cool.  However, studies have found that the expected mean value of such radiation is far below the critical LW flux required to suppress $\htwo$ cooling \citep{Dijkstraetal2008}. At the high energy end, the neutral hydrogen cross section for X-rays and cosmic rays is small, allowing them to easily escape their host minihaloes.  We recently investigated the impact of a cosmic X-ray background on primordial star formation \citep{Hummeletal2014}, finding that the enhanced $\htwo$ cooling provided by X-ray ionisation overcame the additional heating, cooling the gas within the core of the minihalo and lowering the Jeans mass required for collapse; here we focus on the impact of a cosmic ray (CR) background.  

Early work on the impact of CRs on primordial star formation focused on the production of ultra-high-energy CRs (UHECRs) by the decay of ultra-heavy X particles \citep{ShchekinovVasiliev2004,VasilievShchekinov2006,RipamontiMapelliFerrara2007}.  With energies above the Greisen--Zatsepin--Kuzmin (GZK) cutoff \citep{Greisen1966,ZatsepinKuzmin1966}, UHECRs interact with the cosmic microwave background (CMB) to produce ionising photons, which in turn enhance the free electron fraction of the gas.  
While there are several possible sources of CRs at high redshifts in addition to heavy particle decay, including primordial black holes, topological defects, supermassive particles, and structure formation shocks \citep{BiermannSigl2001,Stanev2004,Pfrommeretal2006}, the most likely soruce of CRs in the early universe is supernova (SN) explosions \citep[e.g.,][]{GinzburgSyrovatskii1969}, wherein CRs are accelerated byt the SN shock wave via the first-order Fermi process.  In this scenario, high-energy particles diffuse back and forth across the shock wave, increasing their energy by a small percentage each time, and resulting in a differential spectrum of CR number density per energy \citep{Longair1994}.
Recent work has focused on the direct collisional ionisation of neutral hydrogen by SN shock-generated CRs, using one-zone models to investigate the impact of a CR background on the chemical and thermal evolution of the gas in a minihalo \citep{StacyBromm2007,JascheCiardiEnsslin2007}.  These studies found that the presence of a CR background enhanced molecular hydrogen formation in the minihalo, cooling the gas and lowering the Jeans mass, and by exension, the charachteristic mass of the stars formed. 

Finally, it is worth noting that CR production in the early universe can be constrained using measurements of the $^6{\rm Li}$ abundance measured in metal-poor stars in the Galactic halo.  While big bang nucleosynthesis  predicts a number ration of $^6{\rm Li/H} \simeq 10^{-14}$, the observed  ratio is   $^6{\rm Li/H} \simeq 10^{-11}$ \citep{Asplundetal2006}. \citet{RollindeVangioniOlive2005} and \citet{RollindeVangioniOlive2006} show that this discrepancy can be explained by pre-Galactic CR nucleosynthesis, thus placing an upper limit on the CR energy density at high redshifts.


This paper is organized as follows: In \RefSec{context} we provide the cosmological context for this study, estimating the expected intensity of the CR background. Our numerical methodology is described in \RefSec{methods}, while our results are found in \RefSec{results}.  Finally, our conclusions are gathered in \RefSec{conclusions}. Throughout this paper we adopt a $\Lambda$CDM model of hierarchical structure formation, using the following cosmological parameters: $\Omega_{\Lambda} = 0.7$, $\Omega_{\rm m} = 0.3$, $\Omega_{\rm B} = 0.04$, and $H_0 = 70 \kms \Mpc^{-1}$.