\section{Introduction}
\label{intro}

The formation of the first stars marks a crucial turning point in cosmic history, signalling the end of the dark ages and initiating a significant increase in the complexity of cosmic evolution.  Understanding the formation of these so-called Population III (Pop III) stars is a crucial aspect of modern astrophysics and cosmology, as their emergence ushers in a fundamental transformation of the early universe \citep{BarkanaLoeb2001,Miralda-Escude2003,Brommetal2009,Loeb2010, Bromm2013}. The heavy elements forged in their cores and dispersed during the violent supernova explosions they produce initiated the chemical enrichment process(\citealt{MadauFerraraRees2001, MoriFerraraMadau2002, BrommYoshidaHernquist2003, Hegeretal2003, UmedaNomoto2003, TornatoreFerraraSchneider2007, Greifetal2007, Greifetal2010, WiseAbel2008, Maioetal2011}; recently reviewed in \citealt{KarlssonBrommHawthorn2013}), while the ionizing radiation and cosmic rays produced during their lives and subsequent deaths began the process of reionization \citep{Kitayamaetal2004, Sokasianetal2004, WhalenAbelNorman2004, AlvarezBrommShapiro2006, JohnsonGreifBromm2007, Robertsonetal2010}.