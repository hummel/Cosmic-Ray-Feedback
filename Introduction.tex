\section{Introduction}
\label{intro}

Cosmic rays (CRs) have long been known to play an important role in the complex physics and chemistry of the interstellar medium (ISM) in local galaxies (\citealt{SpitzerTomasko1968,SpitzerScott1969,GlassgoldLanger1973,GoldsmithLanger1978,CravensDalgarno1978,MannheimSchlickeiser1994,Tielens2005}; recently reviewed by \citealt{StrongMoskalenkoPtuskin2007,GrenierBlackStrong2015}).  
They are an effective source of ionisation and heating in various environments, from preheating the primordial intergalactic medium  \citep[IGM;][]{SazonovSunyaev2015}, to driving outflows in the diffuse ISM \citep[e.g.,][]{Ensslinetal2007,Jubelgasetal2008,SalemBryan2014,Hanaszetal2013,Boothetal2013,SalemBryanHummels2014}, to providing an important (often dominant) source of heating and ionisation in deeply embedded gas clouds and protostellar discs \citep{Spitzer1978,DalgarnoYanLiu1999,IndrioloFieldsMcCall2009,PadovaniGalliGlassgold2009,GlassgoldGalliPadovani2012,PadovaniHennebelleGalli2013,Padovanietal2015}. 

CRs are particularly interesting in the context of Population III (Pop III) star formation, as they provide a continually replenished source of free electrons, enhancing the formation of molecular hydrogen---the only coolant available in primordial gas \citep{LeppShull1984,Abeletal1997,GalliPalla1998,BrommCoppiLarson2002,GloverAbel2008}.  
This enhances the ability of the gas to cool, modifying the characteristic density and temperature at which runaway gravitational collapse sets in, and possibly the characteristic mass of the stars thus formed.   
The characteristic mass of Pop III is critical, as it largely controls the extent to which the first stars influence their environment, determining both their total luminosity and ionising radiation output \citep{Schaerer2002}, in addition to the details of their eventual demise \citep{Hegeretal2003,HegerWoosley2010,MaederMeynet2012}. 
As such, a thorough understanding of how the very first stars impact subsequent episodes of metal-free star formation---sometimes referred to as Pop III.1 and Pop III.2, respectively \citep{McKeeTan2008}---is crucial to developing a comprehensive picture of cosmic evolution.

In the absence of any feedback, pioneering numerical studies suggested that the very first stars were quite massive---on the order of $100\msun$ \citep[e.g.,][]{BrommCoppiLarson1999,BrommCoppiLarson2002,AbelBryanNorman2002,Yoshidaetal2003,BrommLarson2004,Yoshidaetal2006,OSheaNorman2007}. 
However, more recent simulations, aided by increased resolution, have found that significant fragmentation occurs during the star formation process \citep{TurkAbelOShea2009,StacyGreifBromm2010,Clarketal2011a,Clarketal2011b,Greifetal2011,Greifetal2012,StacyBromm2013,Hiranoetal2014,Hosokawaetal2015}, leading to the emerging consensus that the Pop III initial mass function (IMF) was somewhat top-heavy with a characteristic mass of $\sim$ a few $\times 10\msun$ \citep{Bromm2013}. 
While the contribution these stars make to chemical enrichment (\citealt{MadauFerraraRees2001,MoriFerraraMadau2002,BrommYoshidaHernquist2003,MackeyBrommHernquist2003,Hegeretal2003,UmedaNomoto2003,BrommLarson2004,KitayamaYoshida2005,TornatoreFerraraSchneider2007,Greifetal2007,Greifetal2010,WiseAbel2008,Maioetal2011}; recently reviewed in \citealt{Whalenetal2008,Joggerstetal2010,KarlssonBrommHawthorn2013}) and  reionisation \citep{Kitayamaetal2004,Sokasianetal2004,WhalenAbelNorman2004,AlvarezBrommShapiro2006,JohnsonGreifBromm2007,Robertsonetal2010} have been well studied, the consequences for Pop III stars forming in neighbouring minihaloes have been less thoroughly explored.

As Pop III stars form in a predominantly neutral medium, the majority of their ionising output is absorbed, allowing only radiation less energetic than the Lyman-$\alpha$ transition to escape the immediate vicinity of the star-forming halo.  
While far-ultraviolet radiation in the Lyman-Werner (LW) bands ($11.2 - 13.6\ev$) lacks sufficient energy to interact with atomic hydrogen, it can still effectively dissociate molecular hydrogen.
Similarly, photons capable of dissociating $H^-$ limit the ability of the gas to cool by eliminating the $H^-$ channel for producing $\htwo$ \citep{Agarwaletal2012,Agarwaletal2016}.
However, studies have found that the expected mean value of such LW radiation is far below the critical flux required to suppress $\htwo$ cooling \mbox{\citep{Dijkstraetal2008}}; moreover, not all LW photons produced escape their host halo to contribute to this background \citep{Schaueretal2015}.
At the high energy end, the neutral hydrogen cross section for X-rays and cosmic rays is small, allowing them to easily escape their host minihaloes. 
We recently investigated the impact of a cosmic X-ray background generated by high-mass X-ray binaries on primordial star formation \citep{Hummeletal2015}; here we focus on the impact of a CR background consisting of particles accelerated in supernova shock waves via the first-order Fermi process (see Section \ref{sec:context}).  

While the uncertainties involved in estimating the strength of the high-$z$  CR background are quite large,  measurements of the $^6{\rm Li}$ abundance from metal-poor stars in the Galactic halo provide a useful constraint: the observed abundance of  $^6{\rm Li}$ is approximately 1000 times higher than predicted by big bang nucleosynthesis \citep{Asplundetal2006}, strong evidence for the existence of a CR spallation channel to provide a $^6{\rm Li}$ bedrock abundance prior to the bulk of star formation \citep{RollindeVangioniOlive2005,RollindeVangioniOlive2006}. 
Production of this first pervasive CR background by shock-acceleration in Pop III supernovae dovetails nicely with the upper limits this places on the CR energy density at high redshifts \citep{RollindeVangioniOlive2006}.

Early studies of the impact of CRs on primordial star formation focused on the production of ultra-high-energy CRs (UHECRs) by the decay of ultra-heavy X particles \citep{ShchekinovVasiliev2004,VasilievShchekinov2006,RipamontiMapelliFerrara2007}.  
With energies above the Greisen--Zatsepin--Kuzmin (GZK) cutoff \citep{Greisen1966,ZatsepinKuzmin1966}, UHECRs interact with the cosmic microwave background (CMB) to produce ionising photons, which in turn enhance the free electron fraction of the gas.  
Other work investigated the direct collisional ionisation of neutral hydrogen by SN shock-generated CRs, using one-zone models to determine the impact of a CR background on the chemical and thermal evolution of the gas in a minihalo \citep{StacyBromm2007,JascheCiardiEnsslin2007}.  
These studies found that the presence of a CR background enhanced molecular hydrogen formation in the minihalo, cooling the gas and lowering the Jeans mass, and by extension, the characteristic mass of the stars formed. 
We expand upon these one-zone models, using three-dimensional \textit{ab initio} cosmological simulations to investigate the impact of a CR background on Pop III stars forming in a minihalo.

This paper is organized as follows: In Section \ref{sec:context} we provide the cosmological context for this study, estimating the expected intensity of the CR background. Our numerical methodology is described in Section \ref{sec:methods}, while our results are presented in Section \ref{sec:results}.  
Finally, our conclusions are gathered in Section \ref{conclusions}. Throughout this paper we adopt a $\Lambda$CDM model of hierarchical structure formation, using the following cosmological parameters, consistent with the latest measurements from the Planck Collaboration \citep{PlanckParams2015}: $\Omega_{\Lambda} = 0.7$, $\Omega_{\rm m} = 0.3$, $\Omega_{\rm B} = 0.04$, and $H_0 = 70 \kms \Mpc^{-1}$.