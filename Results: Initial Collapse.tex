\section{Results}
\label{sec:results}
\subsection{Initial Collapse}
\subsubsection{CR-enhanced H$_2$ Cooling}
\label{sec:initial_collapse}

The gas in both Halo 1 and Halo 2 invariably collapses to high densities, even as we vary the CR background strength by five orders of magnitude. As seen in Figure \ref{fig:temp}, the collapse occurs in accordance with the standard picture of Pop III star formation, albeit slightly modified by the presence of a CR background.  In each case, the gas heats adiabatically as it collapses until reaching $\sim10^3\kelvin$, at which point $\htwo$ cooling activates.  This allows the gas to cool to $\sim200\kelvin$, whereupon it enters a `loitering' phase \citep{BrommCoppiLarson2002}, increasing in density quasi-hydrostatically until sufficient mass accumulates to trigger the Jeans instability, typically between $10^4\cc$ and $10^6\cc$. Runaway collapse then proceeds until three-body reactions become important at $n\sim10^8\cc$, turning the gas fully molecular by $n\sim10^{12}\cc$, at which point we form sink particles.

As $\ucr$ increases, the resulting ionsation boosts the free $\elec$ fraction.  This in turn improves the efficiency of $\htwo$ formation, enhancing cooling and allowing gas in the loitering phase to reach progressively lower temperatures.  This is evident from Figure \ref{fig:efrac}, where we have shown the average free $\elec$ fraction and gas temperature as a function of density for each simulation. 