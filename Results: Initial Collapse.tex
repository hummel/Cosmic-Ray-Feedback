\section{Results}
\label{sec:results}
\subsection{Minihalo Formation}
\subsubsection{CR-enhanced \htwo Cooling}
\label{sec:initial_collapse}

Figure \ref{fig:temp} shows the gas in the minihalo collapsing to high densities, as seen just prior to the formation of the first sink particle at $n=10^{12}\cc$.  Collapse occures in accordance with the standard picture of Pop III star formation, albeit slightly modified by the presence of a CR background.  In each case, the gas heats adiabatically as it collapses until reaching $\sim10^3\kelvin$, at which point $\htwo$ cooling activates.  This allows the gas to cool to $\sim200\kelvin$, whereupon it enters a `loitering' phase, increasing in density quasi-hydrostatically until sufficient mass accumulates to trigger runaway collapse.  As $\ucr$ increases, the enhanced ionsation rate boosts the free $\elec$ fraction.  This in turn improves the efficiency of $\htwo$ formation, enhancing cooling and allowing gas in the loitering phase to reach progressively lower temperatures.  This is evident from Figure \ref{fig:phase_lines}, where we have shown the average free $\elec$ fraction and gas temperature as a function of density for each simulation.

The CR background also works to heat the gas.  This, in concert with the enhancement in $\htwo$ cooling efficiency, lowers the threshold for efficient cooling and allows the gas to collapse at slightly lowere densities and temperatures.  This is especially evident in the $\ucr = 10^5 u_0$ case, where CR heating allows $\htwo$ cooling to activate at densities nearly an order of magnitude lower than usual.  One important consequence of this is evident from Figure \ref{fig:collapse}, where we show the total gas mass within $1\pc$ (physical) of the minihalo's center as a fuction of redshift.  Defining the center as the average position of the 100 densest particles, we see that the minihalo proceeds to runaway collapse at successively earlier times as $\ucr$ increases.  This has the added effect of diminishing the total amount of cold gas available for star formation, as the minihalo has less time to accrete material before collapsing.
