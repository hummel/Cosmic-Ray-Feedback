\section{Results}
\label{sec:results}
\subsection{Initial Collapse}
\subsubsection{CR-enhanced H$_2$ Cooling}
\label{sec:initial_collapse}

The gas in both Halo 1 and Halo 2 invariably collapses to high densities, even as we vary the CR background strength by five orders of magnitude. As seen in Figure \ref{fig:temp}, the collapse occurs in accordance with the standard picture of Pop III star formation \citep[e.g.,][]{Greifetal2012,StacyBromm2013,Hiranoetal2014,Hosokawaetal2015}, albeit slightly modified by the presence of a CR background.  In each case, the gas heats adiabatically as it collapses until reaching $\sim10^3\kelvin$, at which point $\htwo$ cooling activates.  This allows the gas to cool to $\sim200\kelvin$, whereupon it enters a `loitering' phase \citep{BrommCoppiLarson2002}, increasing in density quasi-hydrostatically until sufficient mass accumulates to trigger the Jeans instability, typically between $10^4\cc$ and $10^6\cc$. Runaway collapse then proceeds until three-body reactions become important at $n\sim10^8\cc$;this turns the gas fully molecular by $n\sim10^{12}\cc$, at which point we form sink particles.

As $\ucr$ increases, the resulting ionsation boosts the free $\elec$ fraction.  This in turn improves the efficiency of $\htwo$ formation, enhancing cooling and allowing gas in the loitering phase to reach progressively lower temperatures; in the $10^5\,u_0$ case, the additional cooling is sufficient to cool the gas all the way to the CMB floor. However, as the gas exits the loitering phase and proceeds to higher densities the CR optical depth increases significantly. In concert with the onset of three-body processes this renders CR ionisation and heating negligible at high densities. As a result, the thermal behaviour of the gas increasingly resembles that seen in the $\ucr=0$ case as the collapse proceeds. To wit, by the time we form sink particles at  $10^{12}\cc$ the thermodynamic state of the gas is remarkably similar, even as we vary the CR background strength by five orders of magnitude. This convergence under a wide range of environmental conditions is similar to that seen in the presence of both an X-ray background \citep{Hummeletal2015} and DM--baryon streaming \citep{StacyBrommLoeb2011a,Greifetal2011b}.

%This is evident from Figure \ref{fig:efrac}, where we have shown the average free $\elec$ fraction and gas temperature as a function of density for each simulation. 