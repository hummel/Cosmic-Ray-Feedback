\subsection{Sink Particles}
\label{sinkParticles}
Our sink particle method is described in \citet{StacyGreifBromm2010}. When a gas particle exceeds $n_{\rm max} = 10^{12}\cc$, it and all non-rotationally-supported particles within the accretion radius are replaced by a single sink particle.  We set $r_{\rm acc}$ equal to the resolution length of the simulation: $r_{\rm acc} = L_{\rm res} \simeq 50\au$, where 
\begin{equation}
L_{\rm res} \simeq 0.5 \left( \frac{M_{\rm res}}{\rho_{\rm max}} \right)^{1/3},
\end{equation}
and $\rho_{\rm max} = n_{\rm max} m_{\rm H}$.  Upon creation, the sink immediately accretes the majority of the particles within its smoothing kernel, resulting in an initial mass for the sink particle $M_{\rm sink}$ close to $M_{\rm res} \simeq 1\msun$.  Following its creation, the density, temperature, and chemical abundances of the sink particle are no longer updated.  The sink's density and temperature are held constant at $10^{12}\cc$ and $650\kelvin$, respectively; the pressure of the sink is set accordingly. Assigning a temperature and pressure to the sink particle allows it to behave as an SPH particle, thus avoiding the creation of an artificial pressure vacuum, which would inflate the accretion rate onto the sink \citep[see][]{BrommCoppiLarson2002, MartelEvansShapiro2006}. Once the sink is formed, additional particles (including smaller sinks) are accreted as they approach within $r_{\rm acc}$ of that sink particle.  After each accretion event, the position and momentum of the sink particle is set to the mass-weighted average of the sink and the accreted particle.