Accounting for this attenuation of the CR background and following the treatment of \citet{StacyBromm2007} and \citet{InayoshiOmukai2011}, the cosmic ray heating rate $\Gamma_{\rm \small CR}$ and ionisation rate $k_{\rm \small CR}$ are given by
\begin{equation}
%\Gamma_{\rm \small CR} = n_{\rm \small H} E_{\rm \small heat} k_{\rm \small CR}.
\Gamma_{\rm \small CR} = 
    \frac{ E_{\rm \small heat}}{50\,{\rm \small eV}} 
    \int_{\epsilon_{\rm min}}^{\epsilon_{\rm max}} 
    - \left| \left( \frac{{\rm d}\epsilon} {{\rm d}t} \right)_{\rm ion} \right|
    \frac{dn_{\rm \tiny CR}}{d\epsilon} e^{-\tau_{\rm \small CR}} d\epsilon,
\end{equation}
and 
\begin{equation}
%k_{\rm \small CR} = \frac{1.82\times10^{-7}\,{\rm \small eV\,s}^{-1}}{50\,{\rm \small eV}} 
%    \int_{\epsilon_{\rm min}}^{\epsilon_{\rm max}} f(\epsilon) \frac{dn_{\rm \tiny CR}}{d\epsilon} d\epsilon,
k_{\rm \small CR} = \frac{\Gamma_{\rm \small CR}}{ n_{\rm \small H} E_{\rm \small heat}}
\end{equation}
where $\epsilon_{\rm min} = 10^6\ev$, $\epsilon_{\rm max}= 10^{15}\ev$, $n_{\rm \small H}$ is the number density of hydrogen, and $E_{\rm \small heat}$ is the energy deposited as heat per interaction \citep{Schlickeiser2002}.
While CRs lose about $50\ev$ per interaction, only about $6\ev$ of that goes towards heating in a neutral medium \citep{SpitzerScott1969, ShullvanSteenberg1985}.

Here we assume the incident CR background is composed solely of protons, and all interactions occur with hydrogen only.  While this neglects the slight difference in average CR energy loss per interaction for hydrogen ($36\ev$; \citet{BakkerSegre1951}) vs helium ($40\ev$; \citet{WeissBernstein1956), the resulting error in the employed heating and ionisation rates is sufficiently small for our purposes, particularly compared to the uncertainties involved in estimating the strength of the CR background.