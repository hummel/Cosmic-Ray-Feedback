\subsection{Cosmic Ray Ionisation and Heating}
\label{CRchem}
To study the impact of an ionising CR background on primordial star formation, we implement a uniform CR background as described in Section \ref{sec:context}. Accounting for attenuation of the CRB as discussed in Section \ref{attenuation}, the cosmic ray heating rate $\Gamma_{\rm \small CR}$ and ionisation rate $k_{\rm \small CR}$ are given by
\begin{equation}
%\Gamma_{\rm \small CR} = n_{\rm \small H} E_{\rm \small heat} k_{\rm \small CR}.
\Gamma_{\rm \small CR} = 
    \frac{ E_{\rm \small heat}}{50\,{\rm \small eV}} 
    \int_{\epsilon_{\rm min}}^{\epsilon_{\rm max}} 
    - \left( \frac{{\rm d}\epsilon} {{\rm d}t} \right)_{\rm ion}
    \frac{dn_{\rm \tiny CR}}{d\epsilon} e^{-\tau_{\rm \small CR}} d\epsilon,
\end{equation}
and 
\begin{equation}
%k_{\rm \small CR} = \frac{1.82\times10^{-7}\,{\rm \small eV\,s}^{-1}}{50\,{\rm \small eV}} 
%    \int_{\epsilon_{\rm min}}^{\epsilon_{\rm max}} f(\epsilon) \frac{dn_{\rm \tiny CR}}{d\epsilon} d\epsilon,
k_{\rm \small CR} = \frac{\Gamma_{\rm \small CR}}{ n_{\rm \small H} E_{\rm \small heat}}
\end{equation}
where $\epsilon_{\rm min} = 10^6\ev$, $\epsilon_{\rm max}= 10^{15}\ev$, $n_{\rm \small H}$ is the number density of hydrogen, and $E_{\rm \small heat}$ is the energy deposited as heat per interaction \citep{Schlickeiser2002}.
While CRs lose about $50\ev$ per interaction, only about $6\ev$ of that goes towards heating in a neutral medium \citep{SpitzerScott1969, ShullvanSteenberg1985}.