\subsection{Chemistry and Thermodynamics}
\label{chemistry}
 We employ the chemistry network described in detail by \citet{Greifetal2009b} and used in \citet{Hummeletal2015}, which follows the abundance evolution of $\h$, $\hplus$, $\hminus$, $\htwo$, $\htwo^+$, $\he$, $\heplus$, $\he^{++}$, $\deut$, $\deut^+$, $\hd$ and e$^-$. All relevant cooling mechanisms are accounted for, including $\h$ and $\he$ collisional excitation and ionisation, recombination, bremsstrahlung and inverse Compton scattering. 
 
 Also included is $\htwo$ cooling induced by collisions with $\h$ and $\he$ atoms and other $\htwo$ molecules.  Three-body reactions involving $\htwo$ become important above $n \gtrsim 10^8\cc$; we employ the intermediate rate from \citet{PallaSalpeterStahler1983}, but see \citet{Turketal2011} for a discussion of the uncertainty of these rates.
In addition, the efficiency of $\htwo$ cooling is reduced above $\sim$$10^9\cc$ as the ro-vibrational lines of $\htwo$ become optically thick above this density.  To account for this we employ the Sobolev approximation together with an escape probability formalism (see \citealt{Yoshidaetal2006, Greifetal2011} for details). 