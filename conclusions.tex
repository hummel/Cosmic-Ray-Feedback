\section{Summary and Conclusions}
\label{conclusions}

We have performed two suites of cosmological simulations employing a range of CR backgrounds spanning 5 orders of magnitude in CR energy density. 
Following the gas evolution as it collapses into a minihalo from IGM densities all the way up to $10^{12}\cc$---by which point three-body processes have turned it fully molecular---allows us to capture the entire impact of CR heating and ionisation on Pop III stars forming in the minihalo via its influence on $\htwo$ and $\hd$ cooling in the gas.

The primary impact of the CR background is to enhance the $\htwo$ cooling efficiency by increasing the ionised fraction of the gas.  
This enables the Rees-Ostriker criterion to be fulfilled sooner, thus expediting minihalo collapse. 
Each simulation thus collapses to high density at a different epoch, with the resulting variation in collapse history and experienced gravitational potential overwhelming any evidence of systematic effects from the CR background on protostellar fragmentation and accretion.

Our first suite of simulations (Halo 1) use the same initial conditions as our prior investigation of the impact of a cosmic X-ray background, allowing for a direct comparison of the results.
CRs are found to be much less efficient than X-rays at heating the gas, such that the collapse suppression observed for a sufficiently strong X-ray background does not occur here.  
While the expedited collapse observed here is also seen in the presence of an X-ray background, the effect is somewhat more pronounced owing to the less efficient nature of CR heating, as there is less additional heating for the enhanced molecular fraction to overcome.

Our second suite of simulations focused on a minihalo collapsing at redshifts more comparable to that of \citet{StacyBromm2007}, in order to provide a more direct comparison.  
While we find similar evidence for a lower gas temperature in the loitering phase, by extending our study beyond densities of $10^6\cc$ we find that in all simulations both the thermodynamic state and fragmentation mass scale of the collapsing gas re-converge with the behaviour observed in the absence of a CR background by the time the gas approaches densities of $10^{12}\cc$.
This suggests the presence of a CR background has little impact on the characteristic mass of the stars formed, in contrast to the conclusions of \citet{StacyBromm2007}.