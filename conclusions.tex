\section{Summary and Conclusions}
\label{conclusions}

We have performed two suites of cosmological simulations employing a range of CR backgrounds spanning five orders of magnitude in energy density. 
We follow the thermodynamic evolution of the gas as it collapses into a minihalo from IGM densities up to $n=10^{12}\cc$, at which point three-body processes have turned the gas fully molecular. 
This allows us to capture the combined impact of CR heating and ionisation on Pop III stars forming in the minihalo via its influence on $\htwo$ and $\hd$ cooling.
Once the gas reaches $n=10^{12}\cc$ we form sink particles and follow the subsequent evolution of the system for $5000\yr$, after which point radiative feedback from the forming protostars can no longer be ignored, and our simulations are terminated.

While it also serves to heat the gas, the primary impact of the CR background is to increase the number of free electrons in the collapsing gas, catalysing the formation of additional molecular hydrogen, which in turn enhances the $\htwo$ cooling efficiency.  
This decreases the cooling time, allowing the gas to fulfil the Rees-Ostriker criterion sooner and expediting minihalo collapse. 
Each simulation therefore collapses to high density at a different epoch, and the resulting variation in experienced gravitational potential causes the collapse history---and thus the accretion disk structure---to differ greatly between simulations.

Our first suite of simulations (Halo 1) used the same initial conditions as our prior investigation of a cosmic X-ray background \citep{Hummeletal2015}, allowing for a direct comparison of the results.
CRs are much less efficient than X-rays at heating the gas, such that the collapse suppression observed for a sufficiently strong X-ray background does not occur here.  
While expedited collapse is observed in the presence of an X-ray background as well, the effect is somewhat more pronounced here owing to the more efficient nature of CR ionisation, as there is less associated heating for the enhanced molecular fraction to overcome.

Our second suite of simulations focused on a minihalo collapsing at $z\simeq21.5$ in the $\ucr=0$ case in order to allow a more direct comparison to the results of \citet{StacyBromm2007}, and control for the influence of the CMB temperature floor on our results.
While we find similar evidence for a lower gas temperature in the loitering phase, extending our study beyond the $n=10^6\cc$ limit of \citet{StacyBromm2007} revealed that the thermodynamic path of the collapsing gas---and thus, the fragmentation mass scale---begins to converge with that of the $\ucr=0$ case for $n\gtrsim10^6\cc$.  
This convergence is observed in all simulations for both Halo 1 and Halo 2, with the thermodynamic state of the gas becoming nearly independent of $\ucr$ by the time we form sink particles at $n=10^{12}\cc$.
The remarkable similarity in the thermodynamic state of the gas just prior to sink formation suggests the presence of a CR background has little impact on the characteristic mass of the stars formed.  
Indeed, the typical mass of the sink particles formed is quite robust, remaining stable across all simulations at $10-40\msun$, very much in line with the expected mass of the first stars in the absence of radiative feedback. In fact, this robust behaviour is observed under a variety of environmental conditions, including X-ray irradiation \citep{Hummeletal2015} and DM--baryon streaming \citep{StacyBrommLoeb2011a,Greifetal2011b}.

While the neutral impact of the CR background on the thermodynamic state of primordial gas at high densities results in a robust characteristic mass that does not vary with $\ucr$, there is a possibility that the enhanced cooling resulting from CR ionisation may increase the velocity dispersion of the infalling gas \citep{Clarketal2011a}.  
This would result in more turbulence in the centre of the minihalo, suppressing fragmentation and thus increasing the mass of the stars formed.  
While we see no evidence to support this, any such systematic effect could be easily masked by the variations in collapse history between simulations.  
Likewise, the fact that each simulation collapses to high densities at a different epoch in a different gravitational potential well limits our ability to draw conclusions regarding the impact of the CR background on fragmentation in the protostellar accretion disc.

Finally, while the characteristic mass---and thus, ionising output---of Pop III appears to be unaffected by the presence of a CR background, CRs still heat the low-density gas in the IGM, reducing its clumping factor. This may have implications for reionisation and the 21-cm signal \citep{FurlanettoPengBriggs2006, SazonovSunyaev2015}.