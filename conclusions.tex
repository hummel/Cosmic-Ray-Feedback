\section{Summary and Conclusions}
\label{conclusions}

We have performed two suites of cosmological simulations employing a range of CR backgrounds spanning five orders of magnitude in energy density. 
Following the gas evolution as it collapses into a minihalo from IGM densities up to $10^{12}\cc$---by which point three-body processes have turned it fully molecular---allows us to capture the entire impact of CR heating and ionisation on Pop III stars forming in the minihalo via its influence on $\htwo$ and $\hd$ cooling in the gas.
Once the gas reaches $n=10^{12}\cc$ we form sink particles, following the subsequent evolution of the system for $5000\yr$, after which point radiative feedback from the forming protostars can no longer be ignored.

While CRs also heat the gas, the primary impact of the CR background is increase the number  of free electrons in the collapsing gas, catalysing the formation of additional molecular hydrogen, which in turn enhances the $\htwo$ cooling efficiency.  
This decreases the cooling time, allowing the gas to fulfil the Rees-Ostriker criterion sooner, expediting minihalo collapse. 
Each simulation therefore collapses to high density at a different epoch, with the resulting variation in collapse history and experienced gravitational potential overwhelming any evidence of systematic effects from the CR background on protostellar fragmentation and accretion.

Our first suite of simulations (Halo 1) used the same initial conditions as our prior investigation of the impact of a cosmic X-ray background, allowing for a direct comparison of the results.
CRs are much less efficient than X-rays at heating the gas, such that the collapse suppression observed for a sufficiently strong X-ray background does not occur here.  
While expedited collapse is observed in the presence of an X-ray background, the effect is somewhat more pronounced here owing to the more efficient nature of CR ionisation, as there is less associated heating for the enhanced molecular fraction to overcome.

Our second suite of simulations focused on a minihalo collapsing at $z\simeq21.5$ in the $\ucr=0$ case in order to allow a more direct comparison to the results of \citet{StacyBromm2007} and control for the influence of the CMB temperature floor on our results.
While we find similar evidence for a lower gas temperature in the loitering phase, extending our study beyond the $n=10^6\cc$ limit of \citet{StacyBromm2007} reveals that the thermodynamic path of the collapsing gas---and thus, the fragmentation mass scale---begins to converge with that of the $\ucr=0$ case for $n\gtrsim10^6\cc$.  
This convergence is observed in all simulations for both Halo 1 and Halo 2, with the thermodynamic state of the gas becoming nearly independent of $\ucr$ by the time we form sink particles at $n=10^{12}\cc$.
