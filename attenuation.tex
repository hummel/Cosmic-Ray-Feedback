\subsection{Optical Depth Estimation}
\label{attenuation}
Each time a CR proton ionises an H atom, an electron with average energy $\langle E \rangle = 35\ev$ is produced \citep{SpitzerTomasko1968}.  Including the ionization energy of $13.6\ev$, the CR proton loses approximately $50\ev$ per scattering. This necessarily places a limit on how many scatterings a CR proton can undergo before losing all its energy to ionisation, as well as limiting the distance it may travel.  This distance may be described by a penetration depth 
\begin{equation}
    D_p(n, \epsilon) \approx \frac{\beta c \epsilon} {-({\rm d}\epsilon / {\rm d}t)_{\rm ion}}
\end{equation}
where \citep{Schlickeiser2002}
\begin{equation}
    - \left( \frac{{\rm d}\epsilon} {{\rm d}t} \right)_{\rm ion}(n, \epsilon)
    = 1.82\times10^{-7}\,{\rm \small eV\,s}^{-1} n_{\h} f(\epsilon),
\end{equation}
\begin{equation}    
    f(\epsilon) = (1 + 0.0185 \,{\rm ln}\beta )\, \frac{2 \beta^2}{\beta_0^3 + 2 \beta^3}
\end{equation}
and 
\begin{equation}
    \beta =  \sqrt{1 - \left( \frac{\epsilon}{m_{\rm \tiny H}c^2}+1 \right)^{-2}}.
\end{equation}
Here, $m_{\rm \small H}$ is the mass of hydrogen, $c$ is the speed of light, $\beta = v/c$ and $({\rm d}\epsilon / {\rm d}t)_{\rm ion}$ is the rate at which a CR proton loses energy to ionisation. $\beta_0$ is the cutoff below which the interaction between CRs and the gas decreases sharply; we use $\beta_0=0.01$, appropriate for CRs traveling through a neutral IGM \citep{StacyBromm2007}.
As $D_p(n, \epsilon)$ is the the mean free path of CRs of energy $\epsilon$ traveling through a gas with number density $n$, we may define an effective cross-section $\sigma_{CR}(n,\epsilon)$ for the interaction
\begin{equation}
\sigma_{CR}(n,\epsilon) = \frac{1}{n D_p(n, \epsilon)}.
\end{equation}
This in turn allows us to define an optical depth
\begin{equation}
\tau_{CR}(n,\epsilon) = N \sigma_{CR}(n,\epsilon)
\end{equation}
where $N$ is the gas column density.

We estimate $N$ using the same technique described in \citep{Hummeletal2015}, where the column density along the pole and equator is well fit by 
\begin{equation}
{\rm log}_{10}(N_{\rm \small pole}) = 0.5323\, {\rm log_{10}}(n) + 19.64
\end{equation}
and
\begin{equation}
{\rm log}_{10}(N_{\rm \small equator}) = 0.6262\, {\rm log_{10}}(n) + 19.57, 
\end{equation}
respectively, assuming every line of sight within 45 degrees of the pole experiences column density $N_{\rm \small pole}$ while every other line of sight experiences $N_{\rm \small equator}$ \citep{Hosokawaetal2011}. This allows us to calculate an effective optical depth
\begin{equation}
e^{-\tau_{\rm \tiny CR}} = \frac{2 \Omega_{\rm \small pole}}{4\pi} e^{-\sigma_{\nu}^i N_{\rm \small pole}} + \frac{4\pi - 2 \Omega_{\rm \small pole}}{4\pi} e^{-\sigma_{\nu}^i N_{\rm \small eq}},
\end{equation}
where
\begin{equation}
\Omega_{\rm \small pole} = \int_0^{2\pi}{\rm d}\phi \int_0^{\pi/4}{\rm sin}\theta \,{\rm d}\theta = 1.84\,{\rm sr}.
\end{equation}
  
  
  
  
  
  
  
  
  