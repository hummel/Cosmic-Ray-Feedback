\subsection{Column Density Estimation}
Each time a CR proton ionises a $H$ atom, an electron with average energy $\langle E \rangle = 35\ev$ is produced \citep{SpitzerTomasko1968}.  Including the ionization energy of $13.6\ev$, the CR proton loses approximately $50\ev$ per scattering. This necessarily places a limit on how many scatterings a CR proton can undergo before losing all its energy to ionisation, as well as limiting the distance it may travel.  This distance may be described by a penetration depth 
\begin{equation}
    D_p(n, \epsilon) \approx \frac{\beta c \epsilon} {-({\rm d}\epsilon / {\rm d}t)_{\rm ion}}
\end{equation}
where \citep{Schlickeiser2002}
\begin{equation}
    - \left( \frac{{\rm d}\epsilon} {{\rm d}t} \right)_{\rm ion}(n, \epsilon)
    = 1.82\times10^{-7}\,{\rm \small eV\,s}^{-1} f(\epsilon),
\end{equation}
\begin{equation}    
    f(\epsilon) = (1 + 0.0185 \,{\rm ln}\beta )\, \frac{2 \beta^2}{\beta_0^3 + 2 \beta^3}
\end{equation}
and 
\begin{equation}
    \beta =  \sqrt{1 - \left( \frac{\epsilon}{m_{\rm \tiny H}c^2}+1 \right)^{-2}}.
\end{equation}
Here, $m_{\rm \small H}$ is the mass of hydrogen, $c$ is the speed of light, $\beta = v/c$ and $({\rm d}\epsilon / {\rm d}t)_{\rm ion}$ is the rate at which a CR proton loses energy to ionisation. $\beta_0$ is the cutoff below which the interaction between CRs and the gas decreases sharply; we use $\beta_0=0.01$, appropriate for CRs traveling through a neutral IGM \citep{StacyBromm2007}.
As $D_p(n, \epsilon)$ is the inverse of the mean free path of CRs of energy $\epsilon$ traveling through a gas with number density $n$, we may define an effective cross-section $\sigma_{CR}(n,\epsilon)$ for the interaction
\begin{equation}
\sigma_{CR}(n,\epsilon) = \frac{1}{n D_p(n, \epsilon)}.
\end{equation}
This in turn allows us to define an optical depth
\begin{equation}
\tau_{CR}(n,\epsilon) = N \sigma_{CR}(n,\epsilon)
\end{equation}
where $N$ is the gas column density.

In order to estimate $N$, we calculate the column density approaching the center of the accretion disk along both polar and equatorial lines of sight. As shown in Figure 1, the colum density remains essentially constant for the duration of the simulation, and is nearly linear in log-space over several orders of magnitude up to the resolution limit of our simulation
($\sim100\au$). This same linear behaviour can be seen versus density as well, as shown in Figure 2. In addition we notice that for a given gas density, the column density along the pole is roughly a factor of 10 lower than along the equator.  Performing an ordinary least squares fit to the combined data from several snapshots for $n > 10^4\cc$, we find that the column density along the pole and equator is well fit by 
\begin{equation}
{\rm log}_{10}(N_{\rm \small pole}) = 0.5323\, {\rm log_{10}}(n) + 19.64
\end{equation}
and
\begin{equation}
{\rm log}_{10}(N_{\rm \small equator}) = 0.6262\, {\rm log_{10}}(n) + 19.57, 
\end{equation}
respectively.

To account for this difference in optical depth, we assume every line of sight within $45\degree$ of the pole experiences column density $N_{\rm \small pole}$ and every other line of sight experiences $N_{\rm \small equator}$. This allows us to calculate an effective optical depth
\begin{equation}
N_{\rm \small eff}(n) = N_{\rm \small pole}(n) \frac{2 \Omega_{\rm \small pole}}{4\pi} + N_{\rm \small equator}(n) \frac{4\pi - 2 \Omega_{\rm \small pole}}{4\pi},
\end{equation}
where
\begin{equation}
\Omega_{\rm \small pole} = \int_0^{2\pi}{\rm d}\phi \int_0^{\pi/4}{\rm sin}\theta \,{\rm d}\theta.
\end{equation}