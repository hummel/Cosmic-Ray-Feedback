\section{Numerical Methods}
\label{sec:methods}
Using the well-tested $N$-body smoothed particle hydrodynamics code GADGET-2 \citep{Springel2005}, we perform our simulations using the same chemistry network, and sink particle method as described in \citet{Hummeletal2015}; the relevant details are summarized below.  In addition to using the same initial conditions as in \citet{Hummeletal2015} to allow for a direct comparison of the impact of X-rays versus cosmic rays on the primordial gas, we perform a second suite of simulations using new initial conditions.  The minihalo in these simulations collapses at a lower redshift; this allows us to ensure our results are not being influenced by the CMB temperature floor.

\subsection{Initial Setup}
\label{setup}
The simulations are initialised at $z=100$ in a 140 comoving kpc box with periodic boundary conditions. To accelerate structure formation within the simulation box an artificially enhanced normalisation of the power spectrum, $\sigma_8 = 1.4$, was used, but the simulations are otherwise initialised in accordance with a $\Lambda$CDM model of hierarchical structure formation. For a discussion of the validity of this choice, see \citet{StacyGreifBromm2010}. High resolution in these simulations is achieved using as standard hierarchical zoom-in technique, with nested levels of refinement at 40, 30, and 20 kpc (comoving).  Using this technique, the highest resolution SPH particles have a mass $m_{\rm SPH} = 0.015\msun$, yielding a maximum mass resolution $M_{\rm res} \simeq 1.5 N_{\rm neigh} m_{\rm SPH} \lesssim 1\msun$ for the simulation.  Here $N_{\rm neigh} = 32$ is the number of particles used in the SPH smoothing kernel \citep{BateBurkert1997}.
