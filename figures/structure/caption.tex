\label{fig:structure}
Simulation structure for both Halo 1 (top) and Halo 2 (bottom) as seen in the final output, $5000\yr$ after the first sink forms.  Shown is a density projection of the minihalo environment on progressively smaller scales for both the $\ucr=0$ and $\ucr=10^5\,u_0$ simulations, as labeled.  White boxes indicate the region depicted on the next smaller scale.  From left to right: filamentary structure of the cosmic web near the minihalo formation site; minihalo formation at the intersection of several filaments; morphology within the $\sim$$100\pc$ virial radius of the minihalo. The density scale for each level is shown just to the right -- note that the scaling changes from panel to panel. Note how the gas in the $\ucr=10^5\,u_0$ simulations appears smoothed out compared to the $\ucr=0$ simulations.  This is due not only to CR heating impeding the collapse, but also to the fact that CR ionisation helps expedite the collapse of gas above a certain density, leaving less time for low-density structure to form.  As a result, the dynamical environment in which the sink particles form varies significantly between simulations.