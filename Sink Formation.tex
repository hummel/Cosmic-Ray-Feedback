\subsection{Star Formation}
\subsubsection{Protostellar Fragmentation and Growth}
\label{subsec:sink_formation}

Figure \ref{fig:sinks} shows the growth over time of all sink particles formed in our simulations, from formation of the first sink particle to simulation's end $5000\yr$ later when radiative feedback can no longer be ignored. 
The first sink particle forms when the gas in the centre of the minihalo reaches densities of $10^{12}\cc$, and develops an accretion disk within a few hundred years. 
In all but three cases, this disk quickly fragments, forming a binary or small multiple within $500\yr$. 
Sinks that survive longer than a few hundred years without undergoing a merger quickly accrete the surrounding gas, growing to $\sim$ a few solar masses within $500\yr$ and typically reaching between 10 and $40\msun$ by simulation's end.
As is evident from Figure \ref{fig:sinks}, there is no clear trend with $\ucr$ in either protostellar growth or accretion rate.
The $10\,u_0$ Halo 1 simulation and $10^5\,u_0$ Halo 2 simulation form only a sink particle, while the $10^3\,u_0$ Halo 2 simulation steadily form a total of 14 sink particles over the $5000\yr$ period.




%Figure \ref{fig:sink_growth} shows, for each simulation, the growth over time of the following quantities: cold core mass $M_{\rm \small core}$; defined here as all gas with $n \geq 100\cc$, `disk' mass $M_{\rm \small disk}$; defined as gas above $n=10^8\cc$, total mass in sinks $M_{\rm \small sink}$, and the star formation efficiency (defined as $M_{\rm \small sink}/M_{\rm \small core}$).  While the $\ucr=0$ to $10^3u_0$ cases all have roughly the same amount of cold gas available for star formation, the $10^4$ and $10^5u_0$ cases both have roughly 20\% less due to their accelerated collapse (see Figure \ref{fig:collapse}).  We also note that while all simulations first form sinks once $40 - 80\msun$ of $n>10^8\cc$ gas has accumulated, the accretion rate in the $10^4$ and $10^5u_0$ cases is substantially higher than for the more moderate background cases, which all grow their disk mass a roughly the same rate.  The high disk accretion rate experienced under an intense CR background can be attributed to the enhanced $\htwo$ cooling present in these cases.  As the gas exits the loitering phase, it can more easily radiate away its graviational energy, allowing for rapid collapse.

%Once the gas reaches $10^{12}\cc$, sink particles are formed; the growth over time of all individual sink particles is shown in Figure \ref{fig:sink_growth}. The growth history for each $\ucr > 0$ simulation is shown with the $\ucr=0$ sink growth history marked for reference.  We see that the $\ucr = u_0$ to $10^3u_0$ cases all show roughly similar behaviour to $\ucr=0$, with 2-4 sinks forming in each case and comparable growth rates.  We begin to see a departure from this standard behaviour in the $10^4u_0$ case.  While the growth rate of the sinks in this simulation is roughly canonical, twice as many sinks form, totalling 7 by simulation's end.  While the $\ucr=10^4u_0$ case begins to transition away from the canonical behaviour ofthe more moderate background cases, the $10^5u_0$ case represents a clear departure.  Vigorous fragmentation occurs; 5 sinks form within 100 years, and a total of 15 sinks form over the 5000 years we follow the system.  In addition, enduring sinks in this simulation experience rapid growth, until the fragmentation process disrupts their gas supply.

