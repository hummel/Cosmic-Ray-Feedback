\subsection{Star Formation}
\subsubsection{Protostellar Fragmentation and Growth}
\label{subsec:sink_formation}

Figure \ref{fig:sinks} shows the growth over time of all sink particles formed in our simulations, from formation of the first sink particle to simulation's end $5000\yr$ later when radiative feedback can no longer be ignored. 
The first sink particle forms when the gas in the centre of the minihalo reaches densities of $10^{12}\cc$, and develops an accretion disk within a few hundred years. 
In all but three cases, this disk quickly fragments, forming a binary or small multiple within $500\yr$. 
Sinks that survive longer than a few hundred years without undergoing a merger quickly accrete the surrounding gas, growing to $\sim$ a few solar masses within $500\yr$ and typically reaching between 10 and $40\msun$ by simulation's end.

As is evident from Figure \ref{fig:sinks}, there is no clear trend with $\ucr$ in either protostellar growth or accretion rate.
The $10\,u_0$ Halo 1 simulation and $10^5\,u_0$ Halo 2 simulation form only a single sink for example, while the $10^3\,u_0$ Halo 2 simulation steadily forms a total of 14 sink particles over the $5000\yr$ period.
This suggests that final stages of the collapse are influenced more by turbulence than the strength of the CR background, and can be primarily attributed to the wide variety of dynamical environments in which the sink particles form.
This is demonstrated in Figure \ref{fig:disks}, where we show the accretion disk structure within the central $20,000\au$ of each simulation $5000\yr$ after the first sink forms.
The wide variety of accretion disk structures arising from identical initial conditions is a direct consequence of the expedited collapse discussed in Section \ref{sec:expedited_collapse}.