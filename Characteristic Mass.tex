\subsubsection{Characteristic Mass}

While the final stages of the collapse appear to be somewhat chaotic, with protostellar fragmentation driven primarily by small-scale turbulence rather than the CR background, the typical mass of the sinks formed remains quite stable across all simulations at $10-40\msun$.
This is very much in line with the prevailing consensus for the expected mass of the first stars in the absence of any radiative feedback of $\sim$ a few $\times10\msun$ \citep{Bromm2013}.  
As noted in Section \ref{sec:initial_collapse}, the thermodynamic behaviour of the gas displays a remarkable similarity as it approaches sink formation densities, regardless of $\ucr$.
This suggests that the influence of the CR background is restricted to lower densities, with little to no effect on the protostellar cores from which the first stars ultimately form.

The universality of this characteristic mass is further supported by Figure \ref{fig:Mbe}, where we show the average gas temperature as a function of number density, as well as the approximate fragmentation mass scale, as estimated by the Bonnor-Ebert mass \citep[e.g.,][]{StacyBromm2007}:
\begin{equation}
    M_{\rm \small BE} = 700\msun \left(\frac{T}{200\kelvin}\right)^{3/2}
                                 \left(\frac{n}{10^4\cc}   \right)^{-1/2},
\end{equation}
where $n$ and $T$ are the number density and corresponding average temperature.
Approaching sink formation densities $M_{\rm \small BE}$ is nearly independent of $\ucr$, in agreement with the observed lack of evolution in the sink mass across our entire suite of simulation.

Also show in Figure \ref{fig:Mbe} are the one-zone calculations from \citet{StacyBromm2007}, who investigated the impact of a CR background on Pop III star formation in a $z=21$ minihalo using the CR background strengths shown in Figure \ref{fig:ucr}.