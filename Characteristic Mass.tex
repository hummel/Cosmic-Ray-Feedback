\subsubsection{Characteristic Mass}

While the final stages of the collapse appear to be somewhat chaotic, with protostellar fragmentation driven primarily by small-scale turbulence rather than the CR background, the characteristic mass of the sinks formed remains quite stable across all simulations at $10-40\msun$.
This is very much in line with the prevailing consensus for the expected mass of the first stars in the absence of any radiative feedback of $\sim$a few $\times10\msun$ \citep{Bromm2013}.  
As noted in Section \ref{sec:initial_collapse}, the thermodynamic state of the gas as it approaches sink formation densities is remarkably similar across all simulations, regardess of CR background strength.
This suggests that the influence of the CR background is restricted to lower densities and has little to no effect on the protostellar cores from which the first stars ultimately form.

This is demonstrated in Figure \ref{fig:Mbe}, where we show the average gas temperature as a function of radius as well as the resulting estimate of the Bonnor-Ebert mass

This result stands in contrast to the findings of \citet{StacyBromm2007}, 